\section{\textit{Lean Startup}}

Esta metodologia \cite{ries2011lean} propõe uma abordagem contínua e sistemática de inovação sob a perspectiva das \textit{startups}, seu pensamento pode ser resumido em sete princípios: otimizar o todo, eliminar desperdícios, entregar valor, aprender constantemente, entregar rapidamente, engajar todos os participantes e melhorar continuamente \cite{poppendieck2003lean}. 


Para isso, o autor propõe um ciclo de aprendizado contínuo chamado \textit{Build-Measure-Learn}, para melhor compreensão dos desejos e das necessidades dos clientes. A fase de \textit{Build} foca na construção da funcionalidade, enquanto a fase de \textit{Measure} concentra-se na instrumentação dos dados de comportamento do usuário no sistema. Por fim, a fase de \textit{Learning} tem como objetivo usar os dados coletados para entender como as hipóteses e suposições foram realmente recebidas após a \textit{release}, a fim de construir um corpo de conhecimento sobre o produto e fomentar um ambiente de contínua inovação.
