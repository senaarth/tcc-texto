\section{Experimentação na Engenharia de Software}
\label{subsec:experimentacao}

A experimentação é um domínio fundamental da investigação empírica, oferecendo uma abordagem disciplinada e quantificável para a análise de fenômenos de interesse. Na engenharia de software, essa abordagem é crucial, pois trata-se de uma área multidisciplinar que depende diretamente da avaliação da atividade humana. A pesquisa científica se torna especialmente relevante em momentos de tomada de decisão sobre como softwares são desenvolvidos, contribuindo para o avanço do conhecimento por meio da avaliação das atividades realizadas por pessoas em seus diferentes contextos \cite{wohlin_experimentation_2012}.

\citeonline{basili_experimental_1993} apresentou quatro métodos de pesquisa na Engenharia de \textit{Software}: \textit{Scientific}, \textit{Engineering}, \textit{Empirical} e \textit{Analytical}. Dentre eles, o método Empírico (\textit{Empirical}) é amplamente utilizado em ciências sociais e psicologia para estudar o comportamento humano em contextos onde leis formais não se aplicam, sendo igualmente aplicável à Engenharia de \textit{Software} quando se é necessário lidar com aspectos que não são puramente técnicos \cite{wohlin_experimentation_2012}.

Entre as estratégias de investigação empírica está o \textbf{Estudo de Caso}, que utiliza diversas fontes de dados para investigar um fenômeno em seu contexto real. Este método avalia o caso em questão de forma quantitativa ou qualitativa, sem intervenção ou controle, sendo, portanto, um estudo observacional \cite{runeson_case_study_2012}.

Outra estratégia é o \textbf{Experimento}, que, ao contrário do estudo de caso, busca um controle sistemático e direto sobre a situação estudada. Para isto, identifica o contexto de interesse e coleta amostras de variáveis para validar teorias, confirmar ou explorar relacionamentos. É normalmente utilizado quando o objetivo é comparar diferentes tratamentos para uma mesma situação \cite{wohlin_experimentation_2012}.

Utilizada no contexto deste trabalho, há também a \textbf{Revisão Sistemática da Literatura}, que busca reunir evidências empíricas de diversas fontes literárias através de buscas em portais, bancos de dados acadêmicos ou outras fontes. Dessa forma, através do seu protocolo, é possível sintetizar o estado da arte do domínio desejado por meio de uma avaliação sistemática das informações encontradas na literatura \cite{kitchenham_rsl}.

\todo[inline, color=pink]{Salvo engano eu havia feito um comentário nesse semtido. Essa introdução ficou restrita. Por mais que você detalhe os métodos que você vai utilizar, nessa introdução, precisa apararecer uma visão geral. Tipo o cap.2  do Wohlin. Por exemplo, flatou falar de survey/questionário, pesquisa-ação, estudos etnográficos...}


\subsection{Experimento}

Este método de investigação permite a exploração e validação de relacionamentos através de testes de hipótese e análises estatísticas. Seguindo o apresentado por \citeonline{wohlin_experimentation_2012}, serão apresentados os principais artefatos envolvidos neste processo, bem como as etapas necessárias para sua realização.

\subsubsection{Artefatos}

\begin{itemize}
    \item \textbf{Variáveis}: ao conduzir um experimento, se observa o comportamento de determinadas variáveis durante o processo escolhido. Aquelas que são alteradas para gerar os resultados são chamadas de \textbf{variáveis independentes}, ou \textbf{fatores}. Já aquelas que são responsáveis por informar os efeitos encontrados, são as \textbf{variáveis dependentes};
    \item \textbf{Tratamentos}: valor particular e específico aplicado a um fator. Cada tratamento representa uma configuração particular que é testada no experimento;
    \item \textbf{Objetos}: são os elementos ou itens sobre os quais os tratamentos são aplicados e avaliados, ou seja, quaisquer artefatos ou processos que estejam sendo estudados ou testados, e
    \item \textbf{Sujeitos}: participantes do experimento. Aqueles que são expostos aos diferentes tratamentos.
\end{itemize}

\subsubsection{Processo}

\begin{itemize}
    \item \textbf{Definição de Escopo:} onde se define claramente o problema, os objetivos e as metas do estudo. É crucial estabelecer o escopo para direcionar adequadamente o experimento, o que inclui a formulação inicial da hipótese, a definição do objeto de estudo, o propósito, o foco de qualidade, a perspectiva e o seu contexto;
    \item \textbf{Planejamento:} nesta fase são elaborados os detalhes necessários para a execução do estudo. A hipótese a ser testada é formalizada, são indetificadas as variáveis independentes e dependentes, escolhe-se o \textit{design} do experimento e são preparados os instrumentos necessários. É importante determinar os valores das variáveis e preparar os objetos de estudo, assim como desenvolver diretrizes para a coleta de dados e avaliar a validade dos resultados esperados;
    \item \textbf{Operação:} condução efetiva do experimento. Esta fase inclui a preparação dos sujeitos e materiais, a execução do experimento conforme o planejamento e a validação dos dados coletados para garantir que são corretos e representativos. A execução deve seguir o plano estabelecido para assegurar a integridade dos resultados;
    \item \textbf{Análise e Interpretação:} envolve analisar e interpretar os dados coletados. Utiliza-se estatísticas descritivas para compreender os dados e realiza-se testes de hipótese para avaliar a relação entre as variáveis. A análise pode incluir a redução de dados e a interpretação dos resultados, ajudando a validar as conclusões do experimento;
    \item \textbf{Apresentação e Empacotamento:} concentra-se na documentação e apresentação dos resultados do experimento. O objetivo é comunicar os achados de forma clara e eficaz, o que pode incluir a preparação de relatórios, a publicação dos resultados e a criação de pacotes de laboratório para facilitar a replicação do estudo. Uma documentação completa e detalhada é essencial para garantir que os resultados possam ser replicados e utilizados para futuras pesquisas.
\end{itemize}


\subsection{Estudo de Caso}

Permite uma compreensão aprofundada e contextualizada dos fenômenos de interesse através de investigações dentro de seus contextos reais. Além disso, é flexível, o que possibilita ajustes iterativos em diferentes momentos do processo. A seguir, serão definidas suas etapas de realização: definição, preparação e coleta de dados, análise de dados e relato dos resultados \cite{yin_case_study_2009}.

\subsubsection{Definição}

Segundo \citeonline{wohlin_experimentation_2012}, é a etapa inicial, nela são definidas as bases da pesquisa, através do levantamento dos seguintes artefatos:

\begin{itemize}
    \item \textbf{Objetivo:} inicialmente, é formulado de forma genérica, com um foco principal que será refinado ao longo do estudo. Define o que o pesquisador pretende alcançar e as perguntas que guiarão a investigação;
    \item \textbf{Caso:} definição do objeto de estudo da pesquisa. Pode ser um projeto, um processo, um grupo de pessoas, ou qualquer outro fenômeno que esteja sendo investigado. O caso deve necessariamente ser observado em seu contexto real, permitindo que o pesquisador capture as complexidades e nuances do ambiente;
    \item \textbf{Trabalhos Relacionados:} resultado de uma revisão sobre o estado geral da área de conhecimento à ser estudada, visando pontos de partida em outros trabalhos. Ajuda a identificar lacunas de pesquisa, construir o referencial teórico e fornecer embasamento para a formulação das questões de pesquisa;
    \item \textbf{Questões de Pesquisa:} devem ser claras e orientadoras. Funcionam como um norte para o estudo, direcionando o processo de coleta e análise de dados. Perguntas bem formuladas garantem que o estudo permaneça focado e relevante, abordando as principais preocupações do fenômeno em análise;
    \item \textbf{Métodos:} abordagem de coletas de dados à serem adotadas, dependem das questões de pesquisa e do contexto. A definição clara dos métodos contribui para a validade e confiabilidade do estudo. E
    \item \textbf{Seleção:} refere-se à escolha das fontes de dados e dos participantes do estudo. É crucial que essas escolhas sejam feitas de maneira consciente, considerando a relevância e representatividade das fontes em relação ao objetivo do estudo.
\end{itemize}

\subsubsection{Preparação e Coleta de Dados}
Aqui, o pesquisador deve estruturar cuidadosamente o processo, garantindo que os dados coletados sejam pertinentes e válidos. \citeonline{wohlin_experimentation_2012} destacam a importância da triangulação, que envolve o uso de múltiplas fontes de dados para fortalecer a validade dos resultados. 

A coleta de dados é uma etapa crítica e pode ocorrer de maneira contínua ao longo do estudo. A combinação de diferentes métodos e fontes de dados fortalece a pesquisa e proporciona uma visão mais abrangente do fenômeno. De acordo com \citeonline{lethbridge2005studying},  as técnicas de coleta de dados são divididas em três graus:

\begin{itemize}
    \item \textbf{Primeiro Grau:} técnicas que exigem contato direto com os participantes, como entrevistas, questionários e grupos de discussão. Essas técnicas fornecem dados ricos e detalhados, mas exigem maior envolvimento do pesquisador e podem ser mais caras e demoradas.
    
    \item \textbf{Segundo Grau:} envolve a observação direta do ambiente de estudo sem interação com os participantes. Exemplos incluem a observação do trabalho de uma equipe ou a coleta de métricas de um sistema em funcionamento. Essas técnicas permitem captar o comportamento natural dos participantes, mas podem ser limitadas em termos de profundidade.
    
    \item \textbf{Terceiro Grau:} utiliza dados já existentes, como documentação e registros históricos. Embora menos intrusivos, esses métodos podem ser limitados pela qualidade e relevância dos dados disponíveis. No entanto, são úteis para complementar as técnicas de primeiro e segundo grau.
\end{itemize}

\subsubsection{Análise de Dados}

Processo de interpretação e  busca de padrões nos dados coletados, e que permite ao pesquisador identificar as relações entre as variáveis e os resultados observados \cite{yin_case_study_2009}. Segundo \citeonline{wohlin_experimentation_2012}, a análise de dados pode ser realizada por meio de técnicas qualitativas, quantitativas ou uma combinação de ambas, dependendo da natureza dos dados e dos objetivos da pesquisa.

Para garantir uma análise robusta, \citeonline{wohlin_experimentation_2012} sugerem a adoção de estratégias de categorização dos dados, onde os dados são organizados em categorias ou temas, facilitando a identificação de padrões e a construção de teorias baseadas nas evidências empíricas. O uso de software de análise qualitativa pode ser útil para codificar e categorizar grandes volumes de dados textuais.

Além disso, a análise de dados pode incluir a verificação de validade interna e a validação cruzada dos resultados, assegurando que as conclusões não sejam apenas fruto de coincidências ou de vieses do pesquisador, mas que realmente representem o fenômeno estudado \cite{yin_case_study_2009}.

\subsubsection{Relato}

\citeonline{robson2002real} destaca que o relato é essencial não apenas para comunicar os resultados da pesquisa, mas também para permitir a avaliação da qualidade do estudo por parte de outros pesquisadores, e que, para isto, deve abordar os seguintes aspectos:

\begin{itemize}
\item \textbf{Contexto:} Descrição detalhada do ambiente em que o estudo foi conduzido, incluindo informações sobre os participantes, o período e as condições específicas que podem ter influenciado os resultados. Essa seção deve permitir ao leitor compreender as particularidades do caso estudado, ajudando a contextualizar os resultados.

\item \textbf{Procedimentos:} Explicação das etapas seguidas ao longo do estudo, desde a definição do caso até a coleta e análise dos dados. A clareza na descrição dos procedimentos garante a transparência metodológica, permitindo que outros pesquisadores avaliem a robustez do estudo e, eventualmente, o repliquem.

\item \textbf{Resultados:} Apresentação dos achados de forma estruturada, utilizando ferramentas como tabelas, gráficos e citações diretas dos participantes, quando aplicável, para ilustrar os pontos principais. Deve-se fornecer "instantâneos" relevantes dos dados coletados que sustentem as conclusões apresentadas, garantindo uma cadeia de evidências sólida.

\item \textbf{Discussão:} Interpretação dos resultados à luz da literatura existente, identificando as contribuições do estudo, suas limitações e as implicações práticas ou teóricas. A discussão deve situar os achados no contexto mais amplo do campo de estudo, explorando como eles se relacionam com pesquisas anteriores e quais novas questões surgem a partir das descobertas.

\item \textbf{Conclusões:} Resumo das principais descobertas e sugestões para pesquisas futuras. As conclusões devem sintetizar as contribuições do estudo e indicar como ele avança o conhecimento na área, além de propor direções para estudos subsequentes.
\end{itemize}

