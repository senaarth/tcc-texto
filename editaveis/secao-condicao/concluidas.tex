\section{Atividades Concluídas}

Na Seção \ref{cronograma1}, foram descritas as atividades realizadas durante a primeira etapa desta pesquisa, com o objetivo de detalhar como os objetivos do trabalho seriam alcançados. A Tabela \ref{tab:atividades1} apresenta essas atividades, destacando o status atual de cada uma e os artefatos gerados por suas respectivas execuções.

\begin{table}[]
\centering
    \caption{Condição das Atividades da Primeira Etapa do Trabalho}
    \begin{tabular}{|p{8.5cm}|p{3cm}|p{3cm}|}
        \hline
        \textbf{Atividade} & \textbf{Condição} & \textbf{Resultados} \\ \hline
        Contextualização em Experimentação na Engenharia de Software & Concluída & Seção \ref{subsec:experimentacao} \\ \hline
        Contextualização em Experimentação Contínua & Concluída & Seções \ref{sec:ref-experimentacao-continua} e \ref{sec:ref-analise-dados} \\ \hline
        Definição dos Objetivos e Protocolo de Pesquisa & Concluída & Seções \ref{sec:rsl-protocolo} e \ref{subsec:objetivos-pesquisa} \\ \hline
        Execução da Busca e Seleção do Material & Concluída & Seção \ref{subsec:experimentacao} \\ \hline
        Leitura do Material Selecionado & Concluída & Seção \ref{sec:rsl-resultados} \\ \hline
        Definição da Proposta de Estudo de Caso & Concluída & Capítulo \ref{ch:proposta} \\ \hline
        Escrita da Monografia & Concluída & - \\ \hline
        Revisão da Monografia & Concluída & - \\ \hline
        Apresentação da Monografia & Agendada & - \\ \hline
    \end{tabular}

    \label{tab:atividades1}
    
    \begin{center}
        \text Fonte: Autor
    \end{center}
\end{table}
