% \section{Metodologia}

% \subsection{Classificação Metodológica}


% \begin{figure}[H]
% \centering
% \caption{Titulo da imagem}
% \includegraphics[width=1\textwidth]{figuras/secao-metodologia/Metodologia.png}
% \legend {Colocar aqui a legenda da figura}
% \label{fig:metodologia}
% \end{figure}

% \subsection{Plano Metodológico}
% Conforme Brereton et al (BRERETON et al., 2008), o processo de um estudo de
% caso possui quatro fases principais: Planejamento; Coleta de Dados; Análise de Dados; e
% Relatórios, que compõem o plano metodológico adotado neste trabalho:

% O plano metodológico é baseado em \citeonline{brereton_2008}, que apresenta um \textit{Protocolo de Estudo de Caso}, com vários itens a serem desenvolvidos, e esses distribuídos em quatro grandes fases: Planejamento; Coleta de Dados; Análise de Dados; e Relatórios.

% Essas fases compreendem, de forma resumida:

% \begin{itemize}
%     \item \textbf{Planejamento da Pesquisa:} nessa fase apresenta-se o contexto do pesquisa com a revisão bibliográfica, pergunta de pesquisa,  definição dos objetivos do trabalho e a definição de um plano metodológico, com as escolhas metodológicas;
    
%     \item \textbf{Coleta dos Dados:} essa fase compreende o levantamento e a aplicação das técnicas de coleta de dados como revisão documental, revisão bibliográfica e estudo de caso;
    
%     \item \textbf{Análise dos Dados:} nessa fase realizam-se a interpretação e análise dos dados coletados, assim como a análise da validade do trabalho, cuja meta é avaliar a percepção dos gestores da empresa em relação as diretrizes propostas;
    
%     \item \textbf{Relatório:} por fim, o relatório é constituído pelos resultados deste trabalho e caracterizado por esta monografia.
% \end{itemize}

% O planejamento metodológico do \textit{Protocolo de Estudo de Caso} \cite{brereton_2008} é apresentado no Capítulo Proposta (REVISAR - ALINHAR COM NOVA ESTRUTURA).

