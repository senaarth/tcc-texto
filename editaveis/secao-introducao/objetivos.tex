
\section{Objetivos}
\label{subsec:objetivos-pesquisa}

O principal objetivo deste estudo é investigar como a adoção de práticas de experimentação contínua pode auxiliar na análise da qualidade em uso em um produto de \textit{software} e na tomada de decisão estratégica. Para alcançar esse objetivo geral, foram definidos os seguintes objetivos específicos, que deverão guiar a realização da primeira etapa deste trabalho:

\begin{itemize} 
    \item Realizar um levantamento teórico nas áreas de experimentação, qualidade de \textit{software}, análise estatística  de dados e desenvolvimento orientado a dados; 
    \item Adaptar o processo de desenvolvimento já existente no produto avaliado, incluindo práticas de experimentação contínua para posterior execução e avaliação; 
    \item Planejar um estudo de caso para observar a aplicação do processo proposto no ambiente real de desenvolvimento e uso do produto, objeto desta investigação;
    \item Executar as atividades propostas, coletar métricas de uso e comparar diferentes versões do produto de \textit{software}, avaliando estatisticamente qual delas foi melhor recebida pelos usuários. Ou seja, qual versão apresenta melhor atributo de qualidade em uso percebido. Além disso, documentar e analisar os resultados dos experimentos;
    \item Realizar uma coleta de opinião dos envolvidos no processo em relação às atividades realizadas; e 
    \item Analisar e documentar os principais achados deste estudo. 
\end{itemize}