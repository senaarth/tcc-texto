\section{Contexto}\label{contextualizacao}



A tomada de decisão baseada em evidências é um dos princípios fundamentais de gestão da qualidade definido na norma \citeonline{iso9000}. A ISO9000 destaca como este princípio é essencial para a gestão da qualidade de um produto, contribuindo para a tomada de decisões técnicas e gerenciais eficazes. Já a norma \citeonline{iso25010}, evolução da \citeonline{iso9126}, é específica para avaliação de requisitos de qualidade de software e sistemas. Ela foca nos aspectos da qualidade do produto, que inclui aspectos da visão da qualidade em uso. 

No contexto da caracterização da qualidade de software, os modelos de McCall \cite{mccall1977factors} e de Boehm \cite{boehm1978characteristics} foram pioneiros no fornecimento de visões estruturadas e hierárquicas, influenciando todos os modelos subsequentes. É a partir desta influência que a norma \citeonline{iso25010} descreve os fatores da qualidade de \textit{software}, trazendo uma visão hierárica das características e subcaracterísticas da qualidade em uso.

Dentre as características definidas na visão da  qualidade em uso, destaca-se a eficácia, que representa a acurácia e a completude com as quais os usuários alcançam seus objetivos (vide Figura \ref{fig:quality-in-use}). No entanto, apesar de fornecer essas definições, a ISO não propõe um modelo para medir ou avaliar as características de forma quantitativa, o que impede sua aplicação de forma direta. Assim, se torna necessária a combinação desta com outras teorias e modelos de referência.

Outro modelo de referência existente é a norma \citeonline{nbr9241}, adaptada da \citeonline{iso9241}, que trata das características e subcaracterísticas de usabilidade, dentre elas a eficácia. Essa norma define que, para se avaliar esta característica, é necessária a definição de métricas que possam representar a completude com a qual o produto permite que o usuário realize as tarefas desejadas durante sua utilização.

No contexto de avaliação da qualidade de um produto, uma prática comum é a de coleta de \textit{feedbacks}, ou seja, a coleta de opinião dos usuários e clientes através de entrevistas, enquetes, etre outras práticas. Contudo, esse processo pode ser demorado e muitas vezes não fornecer informações suficientes para a tomada de decisões assertivas \cite{olsson_opinions_2014}. Além disso, com a consolidação da cultura de desenvolvimento orientada à práticas ágeis, de comunidades de \textit{software} livre, do pensamento \textit{Lean}, empresas de tecnologia passaram a adotar práticas que diminuíram o período de disponibilização de \textit{releases} e tornaram essa atividade contínua e frequente \cite{kevic_characterizing_2017}.

A metodologia \textit{Lean} provê um guia para a combinação de \textit{design}, desenvolvimento e validação, agregados em um ciclo de descoberta e entrega de valor \cite{fagerholm_right_2017}. Esta abordagem influenciou o desenvolvimento de \textit{software} e diversas práticas passaram a ser adotadas para que o mesmo se tornasse uma realidade, como a Integração e a Entrega Contínuas \cite{fitzgerald2015continuous}.

Essas e outras práticas se consolidaram no desenvolvimento de \textit{software} de código aberto, com o objetivo de "liberar cedo e frequentemente" \cite{feller2005perspectives}. Esse pensamento se alinha aos princípios \textit{Lean} de aprendizado contínuo e contribuiu para o estabelecimento de práticas que garantem a flexibilidade e rápida adaptação exigidas pelos ambientes ágeis de desenvolvimento \cite{fitzgerald2015continuous}.

Nesse cenário de desenvolvimento contínuo e falta de formalização de mecanismos de coleta de \textit{feedback}, aumenta-se o risco de desalinhamento do produto com as necessidades dos usuários durante a construção de novas funcionalidades \cite{olsson2013data}. E é nessas circunstâncias que surge a chamada Experimentação Contínua, uma abordagem de desenvolvimento que sistematiza a escolha entre duas versões de produto, baseando-se no resultado de testes de hipóteses estatísticas provenientes da coleta de métricas de uso real do \textit{software}.

A Experimentação Contínua tem se tornado um padrão nas grandes empresas de tecnologia \cite{kohavi_seven_2014}. Contudo, a área ainda carece de consenso de uma taxonomia ou corpo de conhecimento bem definido sobre processos, ferramentas, definições ou estratégias \cite{erthal_characterization_2023}.
