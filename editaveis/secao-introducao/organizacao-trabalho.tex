\section{Organização do Trabalho}
\label{sec:organizacao}

Esta subseção visa apresentar a estrutura do documento e o que está presente em cada capítulo desta monografia.

\begin{itemize}
    \item \textbf{Introdução:} apresentação do trabalho, do seu contexto e problemática, além da sua questão de pesquisa e objetivos;
    \item \textbf{Referencial Teórico:} fundamentação teórica do presente trabalho; explanação sobre os principais tópicos relacionados ao contexto desta pesquisa: experimentação em \textit{software}, qualidade de \textit{software}, experimentos controlados, análise de dados e experimentação contínua;
    \item \textbf{Revisão Estruturada da Literatura:} metodologia empregada para a seleção dos estudos que compõem o material bibliográfico; descrição do protocolo utilizado para busca e seleção dos artigos, bem como os resultados desta pesquisa;
    \item \textbf{Proposta de Estudo de Caso:} descrição da estratégia de pesquisa e seu protocolo; definição dos objetivos e da questão de pesquisa; definição e apresentação das atividades a serem realizadas na segunda parte desta monografia; e
    \item \textbf{Condições do Trabalho:} visão geral sobre a situação atual desta investigação; apresentação das atividades já concluídas e daquelas que serão realizadas na segunda etapa desta monografia.
\end{itemize}

\pagebreak