\section{Problema}

Mesmo com o conhecimento de especialistas de negócio e de gerentes de produto, prever as necessidades do usuário é, na maioria das vezes, impreciso \cite{castellion2008do}. Além disso, a coleta direta de opiniões dos usuários também pode não ser assertiva, já que, aquilo que um usuário imagina que deseja muitas vezes difere do que ele realmente utilizaria \textcolor{blue}{\st{na prática}}. Além disso, generalizar os resultados de uma análise qualitativa nem sempre é viável ou ideal \cite{cao2008agile}.

Considerando essa conjuntura, a análise quantitativa por meio de experimentos se apresenta como uma alternativa mais precisa, com a formalização de hipóteses e a definição de métricas-chave para observação \cite{kohavi_oce_and_ab_tests_2017}. No entanto, esse processo exige uma instrumentação adequada para garantir a confiabilidade estatística dos testes realizados. Para isso, é essencial uma definição sistemática do processo de desenvolvimento, incluindo a coleta e análise de dados.

A literatura apresenta diferentes modelos e técnicas para o desenvolvimento orientado a dados, porém carece de uma compreensão compartilhada sobre definições, processos e estratégias, o que dificulta a adoção da prática \cite{quin_b_2024}. Assim, \textbf{a falta de sistematização da coleta e da análise de dados torna a avaliação da qualidade em uso de um produto mais difícil e, muitas vezes, imprecisa}. Esse cenário faz com que a priorização do que será desenvolvido se baseie apenas em opiniões dos envolvidos na construção do produto, ao invés de um processo orientado à tomada de decisão baseada em dados \cite{olsson_opinions_2014}.
