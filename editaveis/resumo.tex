\begin{resumo}

Com o avanço das práticas de entrega contínua, influenciadas pela metodologia \textit{Lean Startup}, as empresas de \textit{software} têm reduzido cada vez mais o tempo entre os lançamentos de novas versões de seus produtos. Em muitas organizações, o \textit{feedback} dos usuários é obtido de forma lenta e pouco sistematizada, o que dificulta uma avaliação precisa do valor entregue pelas funcionalidades desenvolvidas. Nesse contexto, a prática da Experimentação Contínua tem se tornado um padrão nas grandes empresas, utilizando experimentos científicos para validar o produto com base em dados empíricos de uso de suas plataformas. Isso auxilia os gerentes de produto a tomarem decisões fundamentadas em dados, garantindo a qualidade do \textit{software} e o alinhamento com as necessidades reais dos usuários. Considerando esse cenário, este estudo visa compreender as atividades e técnicas necessárias para a sistematização de um processo de Desenvolvimento Orientado a Dados e, com base nessa compreensão, planejar a condução de um Estudo de Caso em uma organização privada brasileira, com o objetivo de aplicar esse processo e observar seu desempenho no apoio à tomada de decisões entre diferentes versões de um produto de \textit{software}.

\vspace{\onelineskip}

\noindent
\textbf{Palavras-chave}: experimentos controlados, experimentação contínua, desenvolvimento orientado a dados, análise de dados, qualidade de software. 
\end{resumo}
