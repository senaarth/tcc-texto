\begin{resumo}[Abstract]
 \begin{otherlanguage*}{english}

With the advancement of continuous delivery practices influenced by the \textit{Lean Startup} methodology, software companies have been shortening their development cycles, releasing new versions of their products at an increasingly rapid pace. In many organizations, the user feedback process is often slow and unsystematic, making it difficult to accurately assess the value delivered by the developed features. In this context, the practice of Continuous Experimentation is becoming a standard in large companies, utilizing controlled experiments to validate products based on empirical usage data. This helps product managers make data-driven decisions, ensuring the quality and alignment of the software with users' actual needs. Considering this scenario, this study aims to understand the activities and techniques necessary to systematize a Data-Driven Development process. Based on this understanding, the study plans to conduct a Case Study in a private Brazilian organization, with the goal of applying this process and observing its performance in supporting decision-making between different versions of a software product.

\vspace{\onelineskip}

\noindent
\textbf{Keywords}: online controlled experiments, continuous experimentation, data-driven development, data analysis, software quality.

 \end{otherlanguage*}
\end{resumo}
